\documentclass{jarticle}
\begin{document}


\section*{Introduction}
These 24 pieces must be percieved as a single ``cycle'' consisting of 12 ``pairs'' of each scale sequence. Each major piece is motivically / thematically related to its parallel minor piece (eg. C major and A major form a ``pair'') and some of the pieces serve as allusions to other compositions. Therefore, the performance of a single piece alone has no meaning. Preferrably all 24 pieces should be performed in the correct order without interruption, or if that is not possible, several pieces should be selected so that there is a logical connection throughout the whole performance. \\




\section*{Notes on Each Piece}
The following notes should not be taken as performance instructions, but rather as background information of how the pieces are constructed and where to find clues for understanding the meaning of each note. \\

\subsection*{1. G Flat Major}
Form: A-B-A-Coda \\
Composition Date: Jan. 1st, 2017 \\

\subsection*{2. E Flat Minor}
Form: A-B-A-Coda \\
Composition Data: Sep. 13th, 2017 \\


The 24-piece cycle starts out with a ``Songs Without Words'' in three voices where the four-note motif (b-ces-es-des) creates connections between each section. In the development section, the motif is exchanged between each voice to build up the tension until the D flat major climax acts as the dominant key and naturally shifts back into G flat major to start the recapitulation. The coda begins in the style of a three-voice imitational counterpoint, a preparation for the succeeding E flat minor. The theme for the canon-like E flat minor is the four-note motif transposed a third lower.  This theme is simultaneously used to develop the piece. The occasional dissonant modulations are imitations of Shostakovich's op. 87. Sustain pedal should not be heavily used throughout and the two pieces should be played at the same tempo i.e. all the four-note motives should be in the same tempo.\\

%24曲のサイクルは三声による「無言歌」で幕を開ける。曲の各部分は4音の動機(b-ces-es-des)によって関連付けられている。展開部では、各声部で動機を交換して属調である変ニ長調の頂点まで盛り上がったのちに、自然に主調へ戻り再現部となる。終結部は3声の模倣対位法により開始し、これは後に続く変ホ短調への準備として働く。カノン風の変ホ短調の主題は4音の動機を三度下に移調したものであり、曲を展開するために継続的に使われる。時折現れる不協和な転調はショスタコーヴィチのop. 87の模倣である。サスティンペダルは踏みすぎないようにすべきであり、2曲は等しいテンポで弾かれるべきである。つまり、4音の動機はすべて同じテンポで奏されることが望ましい。\\


\subsection*{3. D Flat Major}
Form: A-A'-B-A''-A'''-Coda \\
Composition Date: Jan. 6th, 2017 \\

\subsection*{4. B Flat Minor}
Form: A-A'-B-A-Coda \\
Composition Date: Sep. 22nd, 2017 \\


These pieces explore the use of the perfect-fourth interval. In the D flat major, fourth intervals at the higher end of the keyboard is added on the base chord consisting of four perfect fifth intervals. This section depicts the ``oriental'' characteristic of fourths and fifths. In the development section, a new subject is introduced over repeated quarter-note accompaniments. The accessory-like arpeggios concluding this subject resembles the opening motif of the left hand. The latter half of the development is an anti-thesis of the exposition, in that each chord, consisting of two fourth intervals, is individual and there is no wash of sound. On the pedal note of D flat, the chords are now arpeggiated to naturally derive the recapitulation. The first half of the recapitulation repeats the exposition, except that the chords are now arpeggiated. The coda repeats the opening motif on the D major chord. The fierce B flat minor resembles the use of fourths in the genre of rock, a. k. a. power chords and powered fourths. The subject consists only of parallel fourths, which creates harmonies that appear awkward in the classiscal sense. This should appear natural, however, once the listener makes the connection to guitar riffs in pieces such as ``Burn'' or ``Smoke on the Water'' by Deep Purple, or any other of the sort. The pedal markings should be generally obeyed, but the decision should be made by the player with sound effects in consideration. \\





\subsection*{5. A Flat Major}
Form: Minuet (AA-BB-Trio-A-B) \\
Composition Date: Jan. 13th, 2017


\subsection*{6. F Minor}
Form: Theme (8 bars) + 4 Variations \\
Composition Date: Oct. 30th, 2017

The classical structures of minuet and variations are used in these pieces as a `frame' opposed to the rather modern ideas within. In the  A flat major, the minuet section is constructed by the opening subject in time 5, which is usually regarded as a composite beat of 2 and 3. Similarly, the trio section is constructed by the subject in time 7. This second subject can be regarded as 1+6, but this possibility is neglected in the F minor, which includes the first beat as part of the theme. By this method, the use of composite beats as a fundamental unit is explored. In other words, a five-beat phrase or a seven-beat phrase that sounds just as natural as a regular 3-beat phrase or a 4-beat phrase is examined. The theme in the F minor is obviously taken from the trio of the A major. the first two variations are in the same form as the theme with gradual acceleration. The dreamlike third variation imitates the style of Hiromi Uehara and should be played delicately with light pedaling. The final variation is an energetic conclusion which is interrupted by a mysterious sequence of low voiced chords. This phrase is reminiscent of the ``Waldstein'' sonata by Beethoven, which represents  ``briefly looking back to history'' before violently brought back to the present.\\







\end{document}